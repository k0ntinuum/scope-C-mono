

\documentclass{article}
\usepackage[utf8]{inputenc}
\usepackage{setspace}
\usepackage{ mathrsfs }
\usepackage{amssymb} %maths
\usepackage{amsmath} %maths
\usepackage[margin=0.2in]{geometry}
\usepackage{graphicx}
\usepackage{ulem}
\setlength{\parindent}{0pt}
\setlength{\parskip}{10pt}
\usepackage{hyperref}
\usepackage[autostyle]{csquotes}

\usepackage{cancel}
\renewcommand{\i}{\textit}
\renewcommand{\b}{\textbf}
\newcommand{\q}{\enquote}
%\vskip1.0in



\begin{document}

\begin{huge}

{\setstretch{0.0}{
Scope [ C Version Monochrome ]

This program displays the evolution of a neural network and graphs the model against the data being fitted in real time. 

Press and release \b{Q} to quit.

Press and release \b{H} to increase the height of the network.

Press and release \b{L} to increase the length of the network.

Press and release \b{W} to randomize the weights and biases of the network.

Press and release \b{D} to generate new random data points.


Press and release \b{M} to increment the number of data points.

Press and release \b{N} to decrement the number of data points.


Press and release \b{T} to switch all activations to the hyperbolic tangent function.

Press and release \b{S} to switch all activations to sine function.

Press and release \b{Z} for alternating hyperbolic tangent and sine activations.

Press and release \b{I} to use only identity functions (resulting in a linear model.)


Press and release $\leftarrow$ for low rate.

Press and release $\rightarrow$ for high rate.

Press and release $\uparrow$ for max rate.

Press and release $\downarrow$ for zero rate (to pause training).





}}
\end{huge}
\end{document}
